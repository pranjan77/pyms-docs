% chapter02.tex

 %%%%%%%%%%%%%%%%%%%%%%%%%%%%%%%%%%%%%%%%%%%%%%%%%%%%%%%%%%%%%%%%%%%%%%%%%%%%%
 %                                                                           %
 %    PyMS documentation                                                     %
 %    Copyright (C) 2005-2010 Vladimir Likic                                 %
 %                                                                           %
 %    The files in this directory provided under the Creative Commons        %
 %    Attribution-NonCommercial-NoDerivs 2.1 Australia license               %
 %    http://creativecommons.org/licenses/by-nc-nd/2.1/au/                   %
 %    See the file license.txt                                               %
 %                                                                           %
 %%%%%%%%%%%%%%%%%%%%%%%%%%%%%%%%%%%%%%%%%%%%%%%%%%%%%%%%%%%%%%%%%%%%%%%%%%%%%

\chapter{GC-MS Raw Data Model}

\section{\label{sec:chapter2-introuction}Introduction}

PyMS can read gas chromatography-mass spectrometry (GC-MS) data stored in
Analytical Data Interchange for Mass Spectrometry (ANDI-MS),\footnote{ANDI-MS
was developed by the Analytical Instrument Association.} and Joint Committee on
Atomic and Molecular Physical Data (JCAMP-DX)\footnote{JCAMP-DX is maintained by
the International Union of Pure and Applied Chemistry.} formats. These formats
are essentially recommendations, and it is up to individual vendors of mass
spectrometry processing software to implement ``export to ANDI-MS'' or ``export
to JCAMP-DX'' features in their software. It is also possible to get third party
converters. The information contained in the exported data files
can vary significantly, depending on the instrument, vendor's software, or
conversion utility.

For PyMS, the minimum set of assumptions about the information contained in the
data file are:

\begin{itemize}
 \item The data contain the m/z and intensity value pairs across a scan.
 \item Each scan has a retention time.
\end{itemize}

Internally, PyMS stores the raw data from ANDI files or JCAMP files as a
GCMS\_data object.

\section{Reading the raw GC-MS data}

\subsection{Reading JCAMP GC-MS data}

\noindent
[ {\em This example is in pyms-test/20a} ]

The PyMS package pyms.GCMS.IO.JCAMP provides capabilities to read the raw
GC-MS data stored in the JCAMP-DX format.

The file `gc01\_0812\_066.jdx' (located in `data') is a GC-MS experiment
converted from Agilent ChemStation format to JCAMP format using File
Translator Pro.\footnote{ChemSW, Inc.} This file can be loaded in Python
as follows:

\begin{verbatim}
>>> from pyms.GCMS.IO.JCAMP.Function import JCAMP_reader
>>> jcamp_file = "/x/PyMS/data/gc01_0812_066.jdx"
>>> data = JCAMP_reader(jcamp_file)
 -> Reading JCAMP file '/x/PyMS/pyms-data/gc01_0812_066.jdx'
>>>
\end{verbatim}

\noindent
The above command creates the object `data' which is an {\em instance}
of the class GCMS\_data.

\subsection{Reading ANDI GC-MS data}

\noindent
[ {\em This example is in pyms-test/20b} ]

The PyMS package pyms.GCMS.IO.ANDI provides capabilities to read the raw
GC-MS data stored in the ANDI-MS format.

The file `gc01\_0812\_066.cdf' (located in `data') is a GC-MS experiment
converted to ANDI-MS format from Agilent ChemStation (from the same data as in
example 20a above). This file can be loaded as follows:

\begin{verbatim}
>>> from pyms.GCMS.IO.ANDI.Function import ANDI_reader
>>> ANDI_file = "/x/PyMS/data/gc01_0812_066.cdf"
>>> data = ANDI_reader(ANDI_file)
 -> Reading netCDF file '/x/PyMS/pyms-data/gc01_0812_066.cdf'
>>>
\end{verbatim}

\noindent
The above command creates the object `data' which is an {\em instance}
of the class GCMS\_data.

\section{A GCMS\_data object}

\subsection{Methods}

\noindent
[ {\em The following examples are the same in pyms-test/20a and pyms-test/20b} ]

The object `data' (from the two previous examples) stores the raw data as a {\em
GCMS\_data} object. Within the GCMS\_data object, raw data are stored as a list
of {\em Scan} objects and a list of retention times.  There are several methods
available to access data and attributes of the GCMS\_data and Scan objects.

The GCMS\_data object's methods relate to the raw data. The main properties
relate to the masses, retention times and scans. For example, the
minimum and maximum mass from all of the raw data can be returned by the
following:

\begin{verbatim}
>>> data.get_min_mass()
>>> data.get_max_mass()
\end{verbatim}

A list of all retention times can be returned by:

\begin{verbatim}
>>> time = data.get_time_list()
\end{verbatim}

The index of a specific retention time (in seconds) can be returned by:

\begin{verbatim}
>>> data.get_index_at_time(400.0)
\end{verbatim}

\noindent
Note that this returns the index of the retention time in the
data closest to the given retention time of 400.0 seconds.

The method {\tt get\_tic()} returns a total ion chromatogram (TIC) of the data
as an IonChromatogram object:

\begin{verbatim}
>>> tic = data.get_tic()
\end{verbatim}

\noindent
The IonChromatogram object is explained in a later chapter.

\subsection{A Scan data object}

A Scan object contains a list of masses and a corresponding list of intensity
values from a single mass-spectrum scan in the raw data. Typically only
non-zero (or non-threshold) intensities and corresponding masses are stored in
the raw data.

\noindent
[ {\em The following examples are the same in pyms-test/20a and pyms-test/20b} ]

A list of all the raw Scan objects can be returned by:

\begin{verbatim}
>>> scans = data.get_scan_list()
\end{verbatim}

A list of all masses in a scan (e.g. the 1st scan) is returned by:

\begin{verbatim}
>>> scans[0].get_mass_list()
\end{verbatim}

A list of all corresponding intensities in a scan is returned by:

\begin{verbatim}
>>> scans[0].get_intensity_list()
\end{verbatim}

The minimum and maximum mass in an individual scan (e.g. the 1st scan) are
returned by:

\begin{verbatim}
>>> scans[0].get_min_mass()
>>> scans[0].get_max_mass()
\end{verbatim}

\subsection{Exporting data and printing information about a data set}

\noindent
[ {\em This example is in pyms-test/20c} ]

Often it is of interest to find out some basic information about the
data set, e.g. the number of scans, the retention time range, and
m/z range and so on. The GCMS\_data class provides a method info()
that can be used for this purpose.

\begin{verbatim}
>>> from pyms.GCMS.IO.ANDI.Function import ANDI_reader
>>> andi_file = "/x/PyMS/data/gc01_0812_066.cdf"
>>> data = ANDI_reader(andi_file)
 -> Reading netCDF file '/x/PyMS/data/gc01_0812_066.cdf'
>>> data.info()
 Data retention time range: 5.093 min -- 66.795 min
 Time step: 0.375 s (std=0.000 s)
 Number of scans: 9865
 Minimum m/z measured: 50.000
 Maximum m/z measured: 599.900
 Mean number of m/z values per scan: 56
 Median number of m/z values per scan: 40
>>>
\end{verbatim}

To export the entire raw data to a file, use the method write():

\begin{verbatim}
>>> data.write("output/data")
 -> Writing intensities to 'output/data.I.csv'
 -> Writing m/z values to 'output/data.mz.csv'
\end{verbatim}

This method takes the string (``output/data'', in this example)
and writes two CSV files. One has extention ``.I.csv'' and
contains the intensities (``output/data.I.csv'' in this example),
and the other has the extension ``.mz'' and contains the
corresponding table of m/z value (``output/data.mz.csv'' in
this example). In general these are not two-dimensional matrices,
because different scans may have different number of m/z
values recorded.

\subsection{Comparing two GC-MS data sets}

\noindent
[ {\em This example is in pyms-test/20d} ]

Occasionally it is useful to compare two data sets. For example,
one may want to check the consistency between the data set
exported in netCDF format from the manufacturer's software, and
the JCAMP format exported from a third party software.

For example:

\begin{verbatim}
>>> from pyms.GCMS.IO.JCAMP.Function import JCAMP_reader
>>> from pyms.GCMS.IO.ANDI.Function import ANDI_reader
>>> andi_file = "/x/PyMS/data/gc01_0812_066.cdf"
>>> jcamp_file = "/x/PyMS/data/gc01_0812_066.jdx"
>>> data1 = ANDI_reader(andi_file)
 -> Reading netCDF file '/x/PyMS/data/gc01_0812_066.cdf'
>>> data2 = JCAMP_reader(jcamp_file)
 -> Reading JCAMP file '/x/PyMS/data/gc01_0812_066.jdx'
\end{verbatim}

To compare the two data sets:

\begin{verbatim}
>>> from pyms.GCMS.Function import diff
>>> diff(data1,data2)
 Data sets have the same number of time points.
   Time RMSD: 1.80e-13
 Checking for consistency in scan lengths ... OK
 Calculating maximum RMSD for m/z values and intensities ...
   Max m/z RMSD: 1.03e-05
   Max intensity RMSD: 0.00e+00
\end{verbatim}

If the data is not possible to compare, for example because of
different number of scans, or inconsistent number of m/z values
in between two scans, diff() will report the difference.
For example:

\begin{verbatim}
>>> data2.trim(begin=1000,end=2000)
Trimming data to between 1000 and 2000 scans
>>> diff(data1,data2)
 -> The number of retention time points different.
 First data set: 9865 time points
 Second data set: 1001 time points
 Data sets are different.
\end{verbatim}

