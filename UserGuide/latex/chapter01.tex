% chapter01.tex

 %%%%%%%%%%%%%%%%%%%%%%%%%%%%%%%%%%%%%%%%%%%%%%%%%%%%%%%%%%%%%%%%%%%%%%%%%%%%%
 %                                                                           %
 %    PyMS2 documentation                                                     %
 %    Copyright (C) 2005-8 Vladimir Likic                                    %
 %                                                                           %
 %    The files in this directory provided under the Creative Commons        %
 %    Attribution-NonCommercial-NoDerivs 2.1 Australia license               %
 %    http://creativecommons.org/licenses/by-nc-nd/2.1/au/                   %
 %    See the file license.txt                                               %
 %                                                                           %
 %%%%%%%%%%%%%%%%%%%%%%%%%%%%%%%%%%%%%%%%%%%%%%%%%%%%%%%%%%%%%%%%%%%%%%%%%%%%%

\chapter{Introduction}

\section{About PyMS2}

PyMS2 is a Python toolkit for processing of chromatography--mass spectrometry
data. The main idea behind PyMS2 is to provde a framework and a set of
components for rapid development and testing of methods for processing of
chromatography--mass spectrometry data. An important objective of PyMS2 is
to decouple processing methods form visualization and the concept of
interactive processing. This is useful for high-throughput processing tasks
and when there is a need to run calculations in the batch mode.

PyMS2 is modular and consists of several sub-packages written in Python
programming language \cite{python}. PyMS2 is released as open source,
under the GNU Public License version 2.

There are four parts of the pyms project:

\begin{itemize}
  \item pyms2 -- The PyMS2 code
  \item pyms2-docs -- The PyMS2 documentation
  \item pyms2-test -- Examples of PyMS2 use
\end{itemize}

Each part is a separate project on Google Code that can be downloaded
separately. The data used in PyMS2 documentation and examples is available
from the Bio21 Institute server:\\
{\tt http://bioinformatics.bio21.unimelb.edu.au/pyms-data/}\\
In addition, the current PyMS2 API documentation is available from here:\\
{\tt http://bioinformatics.bio21.unimelb.edu.au/pyms.api/index.html}

\section{PyMS2 installation}

There are several ways to install PyMS2 depending your computer configuration
and preferences. The recommended way install PyMS2 is to compile Python
from sources and install PyMS2 within the local Python installation. This
procedure is described below.

PyMS2 has been developed on Linux, and a detailed installation instructions
for Linux are given below. Installation on any Unix-like system should be
similar. We have not tested PyMS2 under Microsoft Windows.

\subsection{Downloading PyMS2 source code}

PyMS2 source code resides on Google Code servers, and can be accessed
from the following URL: http://code.google.com/p/pyms/. Under the
section "Source" one can find the instructions for downloading the
source code. The same page provides the link under "This project's
Subversion repository can be viewed in your web browser" which allows
one to browse the source code on the server without actually downloading
it.

Google Code maintains the source code with the program 'subversion'
(an open-source version control system). To download the source code
one needs to use the subversion client program called 'svn'. The 'svn'
client exists for all mainstream operating systems\footnote{For example,
on Linux CentOS 4 we have installed the RPM package
'subversion-1.3.2-1.rhel4.i386.rpm' to provide us with the subversion
client 'svn'.}, for more information see http://subversion.tigris.org/.
The book about subversion is freely available on-line at
http://svnbook.red-bean.com/. Subversion has extensive functionality.
However only the very basic functionality is needed to download PyMS2
source code.

If the computer is connected to the internet and the subversion client
is installed, the following command will download the latest PyMS2 source
code:

\begin{verbatim}
$ svn checkout http://pyms.googlecode.com/svn/trunk/ pyms
A    pyms/Peak
A    pyms/Peak/__init__.py
A    pyms/Peak/List
A    pyms/Peak/List/__init__.py
.....
Checked out revision 71.
\end{verbatim}

\subsection{PyMS2 installation}

PyMS2 installation consists of placing the PyMS2 code directory (pyms/) in
place visible to Python interpreter.  This can be in the standard place
for 3rd party software (the directory site-packages/). If PyMS2 code is
placed in a non-standard place the Python interpreter needs to be made
aware of it before before it is possible to import PyMS2 modules (see the
Python sys.path.append() command).

We recommend compiling your own Python installation for PyMS2.

In addtion to the PyMS2 core source code, a number of external packages
is used to provide additional functionality. These are explained below.

\subsection{\label{subsec:numpy}Package 'NumPy'}

The package NumPy is provides numerical capabilities to Python. This
package is used throughout PyMS2 (and also required for some external
packages used in PyMS2), to its installation is mandatory. 

The NumPy web site {\tt http://numpy.scipy.org/} provides the installation
instructions and the link to the source code.

\subsection{\label{subsec:pycdf}Package 'pycdf' (required for reading
ANDI-MS files)}

The pycdf (a python interface to Unidata netCDF library) source and
installation instructions can be downloaded from
{\tt http://pysclint.sourceforge.net/pycdf/}. Follow the installation
instructions to install pycdf. 

\subsection{\label{subsec:pycluster}Package 'Pycluster' (required for peak
alignment by dynamic programming)}

The peak alignment by dynamic programming is located in the subpackage
pyms.Peak.List.DPA. This subpackage used the Python package 'Pycluster'
as the clustering engine. Pycluster with its installation instructions
can be found here:
{\tt http://bonsai.ims.u-tokyo.ac.jp/~mdehoon/software/cluster/index.html}.

\subsection{\label{subsec:scipy-ndmage}Package 'scipy.ndimage' (required
for TopHat baseline correction)}

If the full SciPy package is installed the 'ndimage' will be available. However
the SciPy contains large amount of functionality, and its intallation is
somewhat involved. In some situations in may be preferable to install only
the subpackage 'ndimage'. The UrbanSim web site \cite{urbansim} provides
instructions how to install a local copy of 'ndimage'. These instructions
and the link to the file 'ndimage.zip' are here:\\
{\tt http://www.urbansim.org/opus/releases/opus-4-1-1/docs/installation/scipy.html}

\section{Current PyMS2 development environment}

PyMS2 is currently being developed with the following packages:

\begin{verbatim}
Python-2.5.2
numpy-1.1.1
netcdf-4.0
pycdf-0.6-3b
Pycluster-1.41
\end{verbatim}

A quick installation guide for packages required by PyMS2 is given below.

\begin{enumerate}

\item Python installation:

\begin{verbatim}
$ tar xvfz Python-2.5.2.tgz
$ cd Python-2.5.2
$ ./configure
$ make
$ make install
\end{verbatim}

\noindent
This installs python in /usr/local/lib/python2.5.  Make sure that python called
from the command line is the one just compiled and installed.

\item NumPy installation:

\begin{verbatim}
$ tar xvfz numpy-1.1.1.tar.gz
$ cd numpy-1.1.1
$ python setup.py install
\end{verbatim}

\item pycdf installation

Pycdf has two dependencies: the Unidata netcdf library and NumPy. The NumPy
installation is described above. To install pycdf, the netcdf library must
be downloaded ({\tt http://www.unidata.ucar.edu/software/netcdf/index.html}),
compiled and istalled first:

\begin{verbatim}
$ tar xvfz netcdf.tar.gz
$ cd netcdf-4.0
$ ./configure
$ make
$ make install
\end{verbatim}

The last step will create several binary 'libnetcdf*' files in /usr/local/lib.
pycdf can be installed as follows:

\begin{verbatim}
$ tar xvfz pycdf-0.6-3b
$ cd pycdf-0.6-3b
$ python setup.py install
\end{verbatim}

\item Pycluster installation

\begin{verbatim}
$ tar xvfz Pycluster-1.42.tar.gz
$ cd Pycluster-1.42
$ python setup.py install
\end{verbatim}

\item ndimage installation:

\begin{verbatim}
$ unzip ndimage.zip
$ cd ndimage
$ python setup.py install --prefix=/usr/local
\end{verbatim}

\noindent
Since ndimage was installed outside the scipy package, this requires some manual
correction:

\begin{verbatim}
$ cd /usr/local/lib/python2.5/site-packages
$ mkdir scipy
$ touch scipy/__init__.py
$ mv ndimage scipy
\end{verbatim}

\end{enumerate}

\section{Troubleshootings}

The PyMS2 is essentially a python library (a 'package' in python parlance, which
consists of several 'sub-packages'), which for some functionality depends on
other python libraries, such as NumPy, pycdf, and Pycluster. The most likely
problem with PyMS2 installation is a problem with installing one of the PyMS2
dependencies.

\subsection{Pycdf import error}

On Red Hat Linux 5 the SELinux is enabled by default, and this causes the
following error while trying to import properly installed pycdf: 

\begin{verbatim}
$ python
Python 2.5.2 (r252:60911, Nov  5 2008, 16:25:39)
[GCC 4.1.1 20070105 (Red Hat 4.1.1-52)] on linux2
Type "help", "copyright", "credits" or "license" for more information.
>>> import pycdf
Traceback (most recent call last):
  File "<stdin>", line 1, in <module>
  File "/usr/local/lib/python2.5/site-packages/pycdf/__init__.py", line 22, in <module>
    from pycdf import *
  File "/usr/local/lib/python2.5/site-packages/pycdf/pycdf.py", line 1096, in <module>
    import pycdfext as _C
  File "/usr/local/lib/python2.5/site-packages/pycdf/pycdfext.py", line 5, in <module>
    import _pycdfext
ImportError: /usr/local/lib/python2.5/site-packages/pycdf/_pycdfext.so:
    cannot restore segment prot after reloc: Permission denied
\end{verbatim}

This problem is removed simply by disabling SELinux (login as 'root', open the menu
Administration $\rightarrow$ Security Level and Firewall, tab SELinux, change settings
from 'Enforcing' to 'Disabled').

This problem is likely to occur on Red Hat Linux derivative distributions such as CentOS.

