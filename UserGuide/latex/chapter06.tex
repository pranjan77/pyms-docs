% chapter06.tex

 %%%%%%%%%%%%%%%%%%%%%%%%%%%%%%%%%%%%%%%%%%%%%%%%%%%%%%%%%%%%%%%%%%%%%%%%%%%%%
 %                                                                           %
 %    PyMS documentation                                                     %
 %    Copyright (C) 2005-2010 Vladimir Likic                                 %
 %                                                                           %
 %    The files in this directory provided under the Creative Commons        %
 %    Attribution-NonCommercial-NoDerivs 2.1 Australia license               %
 %    http://creativecommons.org/licenses/by-nc-nd/2.1/au/                   %
 %    See the file license.txt                                               %
 %                                                                           %
 %%%%%%%%%%%%%%%%%%%%%%%%%%%%%%%%%%%%%%%%%%%%%%%%%%%%%%%%%%%%%%%%%%%%%%%%%%%%%

\chapter{Peak alignment by dynamic programming}

PyMS provides functions to align GC-MS peaks by dynamic programming
\cite{Robinson07}.  The peak alignment by dynamic programming uses both peak
apex retention time and mass spectra.  This information is determined from the
raw GC-MS data by applying a series of processing steps to produce data that
can then be aligned and used for statistical analysis.  The details are
described in this chapter.

\section{Preparation of multiple experiments for peak alignment
by dynamic programming}

\subsection{Creating an Experiment}

\noindent
[ {\em This example is in pyms-test/60} ]

Before aligning peaks from multiple experiments, the peak objects need to be
created and encapsulated into PyMS experiment objects. During this process
it is often useful to pre-process the peaks in some way, for example to
null certain m/z channels and/or to select a certain retention time range.

To capture the data and related information prior to peak alignment, an
{\tt Experiment} object is used. The {\tt Experiment} object is defined in
{\tt pyms.Experiment.Class}.

The procedure is to proceed as described in the previous chapter. Namely: read
a file; bin the data into fixed mass values; smooth the data; remove the
baseline; deconvolute peaks; filter the peaks; set the mass range; remove
uninformative ions; and estimate peak areas.  The process is given in the
following program listing.

\begin{verbatim}
01  import sys, os
02  sys.path.append("/x/PyMS")
03
04  from pyms.GCMS.IO.ANDI.Function import ANDI_reader
05  from pyms.GCMS.Function import build_intensity_matrix_i
06  from pyms.Noise.SavitzkyGolay import savitzky_golay
07  from pyms.Baseline.TopHat import tophat
08  from pyms.Peak.Class import Peak
09  from pyms.Peak.Function import peak_sum_area
10
11  from pyms.Deconvolution.BillerBiemann.Function import BillerBiemann, \
12      rel_threshold, num_ions_threshold
13
14  # deconvolution and peak list filtering parameters
15  points = 9; scans = 2; n = 3; t = 3000; r = 2;
16
17  andi_file = "/x/PyMS/data/a0806_077.cdf"
18
19  data = ANDI_reader(andi_file)
20
21  # integer mass
22  im = build_intensity_matrix_i(data)
23
24  # get the size of the intensity matrix
25  n_scan, n_mz = im.get_size()
26
27  # smooth data
28  for ii in range(n_mz):
29      ic = im.get_ic_at_index(ii)
30      ic1 = savitzky_golay(ic)
31      ic_smooth = savitzky_golay(ic1)
32      ic_base = tophat(ic_smooth, struct="1.5m")
33      im.set_ic_at_index(ii, ic_base)
34
35  # do peak detection on pre-trimmed data
36
37  # get the list of Peak objects
38  pl = BillerBiemann(im, points, scans)
39
40  # trim by relative intensity
41  apl = rel_threshold(pl, r)
42
43  # trim by threshold
44  peak_list = num_ions_threshold(apl, n, t)
45
46  print "Number of Peaks found:", len(peak_list)
47
48  # ignore TMS ions and set mass range
49  for peak in peak_list:
50      peak.crop_mass(50,540)
51      peak.null_mass(73)
52      peak.null_mass(147)
53      # find area
54      area = peak_sum_area(im, peak)
55      peak.set_area(area)
56
\end{verbatim}

The resulting list of peaks can now be stored as an {\tt Experiment} object.

\begin{verbatim}
from pyms.Experiment.Class import Experiment
from pyms.Experiment.IO import store_expr

# create an experiment
expr = Experiment("a0806_077", peak_list)

# set time range for all experiments
expr.sele_rt_range(["6.5m", "21m"])

store_expr("output/a0806_077.expr", expr)
\end{verbatim}

Once an experiment has been defined, it is possible to limit the peak list to
a desired range using {\tt sele\_rt\_range()}.  The resulting experiment object
can then be stored for later alignment.

\subsection{Multiple Experiments}

\noindent
[ {\em This example is in pyms-test/61a} ]

This example considers the preparation of three GC-MS experiments for peak
alignment. The experiments are named `a0806\_077', `a0806\_078', `a0806\_079',
and represent separate GC-MS sample runs from the same biological sample.

The procedure is the same as above, and repeated for each experiment.  For
example:

\begin{verbatim}
...
# define path to data files
base_path = "/x/PyMS/data/"

# define experiments to process
expr_codes = [ "a0806_077", "a0806_078", "a0806_079" ]

# loop over all experiments
for expr_code in expr_codes:

    print "Processing", expr_code

    # define the names of the peak file and the corresponding ANDI-MS file
    andi_file = os.path.join(base_path, expr_code + ".cdf")
...

...
    # create an experiment
    expr = Experiment(expr_code, peak_list)

    # use same time range for all experiments
    expr.sele_rt_range(["6.5m", "21m"])

    store_expr("output/"+expr_code+".expr", expr)
\end{verbatim}

\noindent
[ {\em This example is in pyms-test/61b} ]

The previous set of data all belong to the same experimental condition.  That
is, they represent one group and any comparison between the data is a within
group comparison. For the original experiment, another set of GC-MS data was
collected for a different experimental condition.  This group must also be
stored as a set of experiments, and can be used for between group comparison.

The experiments are named `a0806\_140', `a0806\_141', `a0806\_142', and are
processed and stored as above (see pyms-test/61b).

\section{Dynamic programming alignment of peak lists from multiple experiments}

\noindent
\begin{itemize}
\item {\em This example is in pyms-test/62}
\item {\em This example uses the subpackage pyms.Peak.List.DPA, which in turn
uses the Python package 'Pycluster'.  For 'Pycluster' installation instructions
see the Section \ref{subsec:pycluster}.}
\end{itemize}

In this example the experiments `a0806\_077', `a0806\_078', and `a0806\_079'
prepared in pyms-test/61a will be aligned, and therefore the script
pyms-test/61a/proc.py must be run first, to create the files
`a0806\_077.expr', `a0806\_078.expr', `a0806\_079.expr' in the directory
pyms-test/61a/output/. These files contain the post-processed peak lists
from the three experiments.

A script for running the dynamic programming alignment on these experiments is
given below.

\begin{verbatim}
"""proc.py
"""

import sys, os
sys.path.append("/x/PyMS/")

from pyms.Experiment.IO import load_expr
from pyms.Peak.List.DPA.Class import PairwiseAlignment
from pyms.Peak.List.DPA.Function import align_with_tree, exprl2alignment

# define the input experiments list
exprA_codes = [ "a0806_077", "a0806_078", "a0806_079" ]

# within replicates alignment parameters
Dw = 2.5  # rt modulation [s]
Gw = 0.30 # gap penalty

# do the alignment
print 'Aligning expt A'
expr_list = []
expr_dir = "../61a/output/"
for expr_code in exprA_codes:
    file_name = os.path.join(expr_dir, expr_code + ".expr")
    expr = load_expr(file_name)
    expr_list.append(expr)
F1 = exprl2alignment(expr_list)
T1 = PairwiseAlignment(F1, Dw, Gw)
A1 = align_with_tree(T1, min_peaks=2)

A1.write_csv('output/rt.csv', 'output/area.csv')
\end{verbatim}


The script reads the experiment files from the directory where they were stored
({\tt 61a/output}), and creates a list of the loaded {\tt Experiment} objects.
Each experiment object is converted into an {\tt Alignment} object with {\tt
exprl2alignment()}. In this example, there is only one experimental condition
so the alignment object is only for within group alignment (this special case
is called 1-alignment). The variable F1 is a Python list containing three
alignment objects.

The pairwise alignment is then performed.  The parameters for the alignment by
dynamic programming are: Dw, the retention time modulation in seconds; and Gw,
the gap penalty.  These parameters are explained in detail in
\cite{Robinson07}.  {\tt PairwiseAlignment()}, defined in {\tt
pyms.Peak.List.DPA.Class} is a class that calculates the similarity between a
ll peaks in one sample with those of another sample.  This is done for all
possible pairwise alignments (2-alignments). The output of
{\tt PairwiseAlignment()} (T1) is an object which contains the dendrogram tree
that maps the similarity relationship between the input 1-alignments, and also
1-alignments themselves.

The function {\tt align\_with\_tree()} takes the object
T1 and aligns the individual alignment objects according to
the guide tree.  In this example, the individual alignments are
three 1-alignments, and the function {\tt align\_with\_tree()} first
creates a 2-alignment from the two most similar 1-alignments
and then adds the third 1-alignment to this to create a
3-alignment. The parameter `{\tt min\_peaks=2}' specifies that any peak
column of the data matrix that has less than two peaks in the final
alignment will be dropped.  This is useful to clean up the data
matrix of accidental peaks that are not truly observed over the
set of replicates.

Finally, the resulting 3-alignment is saved by writing alignment tables
containing peak retention times (`rt1.csv') and the corresponding peak areas
(`area1.csv'). These two files are plain ASCII files is CSV format, and are
saved in the directory {\tt 62/output/}.

\noindent
The file `area1.csv' contains the data matrix where the corresponding peaks are
aligned in the columns and each row corresponds to an experiment. The file
`rt1.csv' is useful for manually inspecting the alignment in some GUI driven
program.

\section{Between-state alignment of peak lists from multiple experiments}

\noindent
[ {\em This example is in pyms-test/63} ]

In the previous example the list of peaks were aligned within a single
experiment with multiple replicates (``within-state alignment'').  In practice,
it is of more interest to compare the two experimental states. In a typical
experimental setup there can be multiple replicate experiments on each
experimental state or condition. To analyze the results of such an experiment
statistically, the list of peaks need to be aligned within each
experimental state and also between the states. The result of such an alignment
would be the data matrix of integrated peak areas. The data matrix contains
 a row for each sample and the number of columns is determined by
the number of unique peaks (metabolites) detected in all the experiments.

In principle, all experiments could be aligned across conditions and
replicates in the one process. However, a more robust approach is to first align
experiments within each set of replicates (within-state
alignment), and then to align the resulting alignments (between-state
alignment) \cite{Robinson07}.

This example demonstrates how the peak lists from two cell states are
aligned. The cell state, A, consisting of three experiments aligned in {\em
pyms-test/61a} (`a0806\_077', `a0806\_078', `a0806\_079') and cell state,
B, consisting of three experiments aligned in {\em pyms-test/61b} (`a0806\_140',
`a0806\_141', and `a0806\_142').

The between group alignment can be performed by the following alignment
commands.

\begin{verbatim}
# between replicates alignment parameters
Db = 10.0 # rt modulation
Gb = 0.30 # gap penalty

print 'Aligning input {1,2}'
T9 = PairwiseAlignment([A1,A2], Db, Gb)
A9 = align_with_tree(T9)

A9.write_csv('output/rt.csv', 'output/area.csv')
\end{verbatim}

\noindent
where A1 and A2 are the results of the within group alignments (as above) for
group A and B, respectively.

In this example the retention time tolerance for between-state alignment is
greater compared to the retention time tolerance for the within-state alignment
as we expect less fidelity in retention times between them.  The same functions
are used for the within-state and between-state alignment. The result of the
alignment is saved to a file as the area and retention time matrices (described
above).

%\section{Comparing two peak lists by using dynamic programming alignment}
%
%The PyMS package {\tt pyms.Peak.List.Metric} provides a function to compare two
%peak lists. This allows peak detection methods from different programs to be
%evaulated or a peak detection method to be compared to a 'known' or expert
%evaluated result.

\section{Common Ion Area Quantitation}
[ {\em This example is in pyms-test/64 } ]
\label{sec:common-ion}

The {\tt area.csv} file produced in the preceding section lists the
total area of each peak in the alignment. The total area is the sum of
the areas of each of the individual ions in the peak. While this approach
produces broadly accurate results, it can result in errors where
neighbouring peaks or unfiltered noise add to the peak in some way.

One alternative to this approach is to pick a single ion which is
common to a particular peak (compound), and to report only the area of this ion
for each occurance of that peak in the alignment. Using the method
{\tt common\_ion()} of the PyMS class Alignment, PyMS can select an ion for
each aligned peak which is both abundant and occurs most often 
for that peak. We call this the 'Common Ion Algorithm' (CIA).

To implement this in PyMS, it is essential that the individual ion
areas have been set (see section \ref{sec:individual-ion-areas}).


\subsection{Using the Common Ion Algorithm}
When using the CIA for area quantitation, a different method of the
class Alignment is used to write the area matrix; {\tt
  write\_common\_ion\_csv()}. This requires a list of the common ions
for each peak in the alignment. This list is generated using the
Alignment class method {\tt common\_ion()}.

Continuing from the previous example, the following invokes common ion
filtering on previously created alignment object 'A9':
\begin{verbatim}
>>> common_ion_list = A9.common_ion()
\end{verbatim}

The variable 'common\_ion\_list' is a list of the common ion for each
peak in the alignment. This list is the same length as the
alignment. To write peak areas using common ion quantitation:

\begin{verbatim} 
>>> A9.write_common_ion_csv('output/area_common_ion.csv',common_ion_list)
\end{verbatim}
