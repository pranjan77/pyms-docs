% chapter09.tex

 %%%%%%%%%%%%%%%%%%%%%%%%%%%%%%%%%%%%%%%%%%%%%%%%%%%%%%%%%%%%%%%%%%%%%%%%%%%%%
 %                                                                           %
 %    PyMS documentation                                                     %
 %    Copyright (C) 2005-2010 Vladimir Likic                                 %
 %                                                                           %
 %    The files in this directory provided under the Creative Commons        %
 %    Attribution-NonCommercial-NoDerivs 2.1 Australia license               %
 %    http://creativecommons.org/licenses/by-nc-nd/2.1/au/                   %
 %    See the file license.txt                                               %
 %                                                                           %
 %%%%%%%%%%%%%%%%%%%%%%%%%%%%%%%%%%%%%%%%%%%%%%%%%%%%%%%%%%%%%%%%%%%%%%%%%%%%%

\chapter{Parallel processing with PyMS}

\section{\label{sec:mpi}Requirements}

Using PyMS parallel capabilities requires installation of the package
'mpi4py', which provides bindings of the Message Passing Interface (MPI)
for the Python programming language. This package can be downloaded
from {\tt http://code.google.com/p/mpi4py/}. Since 'mpi4py' provides
only Python bindings, it requires an MPI implementation. We recommend
using mpich2:\\
{\tt http://www.mcs.anl.gov/research/projects/mpich2/}\\
We show the installation of 'mpich2' and 'mpi2py' on Linux system from
software distributions downloaded from the projects' web site.

\subsection{\label{sec:mpich2}Installation of 'mpich2'}

\begin{enumerate}

\item From the mpich2 project web site download the current distribution of
mpich2 (in our case the file 'mpich2-1.2.1p1.tar.gz').

\item Prepare the directory for mpich2 installation. In this example
we have chosen to use /usr/local/mpich2/. Our version of mpitch2 is
1.2.1, and to allow for the installation of different version later,
we create a subdirectory "1.2.1",

\begin{verbatim}
$ mkdir -vp /usr/local/mpich2/1.2.1
\end{verbatim}

The above command will make the directory /usr/local/mpich2/ and also
/usr/local/mpich2/1.2.1. Note that /usr/local is usually owned by
root, and the above commands may require root privileges.

\item Unpack this file and change to the source code directory:

\begin{verbatim}
$ tar xvfz mpich2-1.2.1p1.tar.gz 
$ cd  mpich2-1.2.1p1
\end{verbatim}

\item Configure, compile, and install mpich2:

\begin{verbatim}
$ ./configure --prefix=/usr/local/mpich2/1.2.1 --enable-sharedlibs=gcc
$ make
$ make install
\end{verbatim}

If /usr/local/mpich2/1.2.1 is owned by rood, the above command
may require root privileges.

\end{enumerate}

\subsection{\label{sec:mpi4py}Installation of 'mpi4py'}

\begin{enumerate}

\item From the mpi4py project web site download the current distribution
of mpi4py (in our case the file 'mpi4py-1.2.1.tar.gz').

\item Unpack this file and change to the source code directory:

\begin{verbatim}
$ tar xvfz mpi4py-1.2.1.tar.gz
$ cd mpi4py-1.2.1
\end{verbatim}

\item Edit the file 'mpi.cfg' to reflect the location of mpich2.  In
our case this file after editing contained the following:

\begin{verbatim}
# MPICH2
[mpi]
mpi_dir              = /usr/local/mpich2/1.2.1
mpicc                = %(mpi_dir)s/bin/mpicc
mpicxx               = %(mpi_dir)s/bin/mpicxx
\end{verbatim}

\item Install mpi4py:

\begin{verbatim}
$ python setup.py install
\end{verbatim}

\item Check that mpi4py works:

\begin{verbatim}
$ python
Python 2.5.2 (r252:60911, Sep 10 2008, 14:39:22) 
[GCC 4.1.1 20070105 (Red Hat 4.1.1-52)] on linux2
Type "help", "copyright", "credits" or "license" for more information.
>>> import mpi4py
>>> 
\end{verbatim}

If the above command import produced no output, mpi4py is installed
properly and ready to use.

\end{enumerate}

\section{\label{sec:parallel-background}Background to using PyMS in parallel}

Any processing that loops through ion chromatograms or mass spectra can
be performed in parallel, by distributing the processing of individual
ion chromatograms or mass spectra to different CPUs by using the
efficient MPI mechanism.

Before the parallel processing can be deployed, data needs to be binned
to produce an IntensityMatrix object, as described in the Section
\ref{sec:intensity-matrix}. This is eessentially a two dimensional
matrix, with ion chromatograms along one dimension and mass spectra
along the other dimension.

Consider the processing which applies a noise smoothing function to
each ion chromatogram. We first read the raw data:

\begin{verbatim}
andi_file = "/x/PyMS/data/gc01_0812_066.cdf"
data = ANDI_reader(andi_file)
\end{verbatim}
 
Then build the intensity matrix, and get its dimensions:

\begin{verbatim}
im = build_intensity_matrix_i(data)
n_scan, n_mz = im.get_size()
\end{verbatim}

The last command sets the variables n\_scan and n\_mz to the number
scans and number of m/z values present in data, respectively.
Processing of ion chromatograms with the noise smoothing function
requires fetching of each ion chromatogram from the data, and
application of the noise smoothing function. This can be achieved 
with a simple loop: 

\begin{verbatim}
for ii in n_mz:
    print ii+1,
    ic = im.get_ic_at_index(ii)
    ic_smooth = window_smooth(ic, window=7)
\end{verbatim}

This example epitomizes the typical processing required on the
GC-MS data. Another, equally important processing, is that of
individual mass spectra. In this case the same logic can be
applied, except that one would loop over the other dimension
of the IntensityMatrix object 'im'. That is, one would loop
over all the scan indices, and use the method 
get\_ms\_at\_index() to fetch individual mass spectra:


\begin{verbatim}
for ii in n_scan:
    print ii+1,
    ms = im.get_ms_at_index(ii)
    # here do something the the mass spectra 'ms'
\end{verbatim}

Processing of data in this fashion is computationally intensive.
A typical data set may consist of 3,000-10,000 scans and ~500
m/z values. If complex processing algorithms are applied to
each ion chromatogram (or mass spectra), the processing will
quickly become computationally prohibitive.

The type of calculation illustrated above is an ideal candidate
for parallelization because each ion chromatogram (or mass
spectrum) are processed independently. PyMS takes advantage
of this and allows one to harvest the power of multiple CPUs
to spead-up the processing. To achieve this PyMS can distributes
the loop from the above (either type, ie. over ion chromatograms
or mass spectra) over the available CPUs, achieveing a linear
speed-up with the number of CPUs.


\section{\label{sec:parallel-pyms}Using PyMS in parallel}

Using PyMS in parallel reqires a minimal intervension, only
that special method of the IntensityMatrix object is invoked
in the for loop described above. For looping over all ion
chromatograms in parallel,

\begin{verbatim}
for ii in im.iter_ic_indices():
    print ii+1,
    ic = im.get_ic_at_index(ii)
    ic_smooth = window_smooth(ic, window=7)
\end{verbatim}

The only change is that 

\begin{verbatim}
for ii in n_mz:
\end{verbatim}

is replaced with

\begin{verbatim}
for ii in im.iter_ic_indices()
\end{verbatim}

The corresponsing method for looping over all mass spectra
would involve replacing:

\begin{verbatim}
for ii in n_scan:
\end{verbatim}

with

\begin{verbatim}
for ii in im.iter_ms_indices()
\end{verbatim}

The special constructs {\tt for ii in im.iter\_ic\_indices():} and
{\tt for ii in im.iter\_ms\_indices()} will distribute the calculation
in parallel if MPI capability is available (ie. mpi4py is installed
on the system, and multiple CPUs are available). If MPI capability
is not available, the processing will be performed in a serial mode. 
Running in parallel also requires some preparations, as explained
in the example below.

Consider how the following script named 'proc.py, that attempts
to perform the noise smoothing example described above, can be
run in serial and parallel mode.

\begin{verbatim}
"""proc.py
"""

import sys
sys.path.append("/x/PyMS")

from pyms.GCMS.IO.ANDI.Function import ANDI_reader
from pyms.GCMS.Function import build_intensity_matrix_i
from pyms.Noise.Window import window_smooth

# read the raw data as a GCMS_data object
andi_file = "/x/PyMS/data/gc01_0812_066.cdf"
data = ANDI_reader(andi_file)

# build the intensity matrix
im = build_intensity_matrix_i(data)

# get the size of the intensity matrix
n_scan, n_mz = im.get_size()
print "Size of the intensity matrix is (n_scans, n_mz):", n_scan, n_mz

# loop over all m/z values, fetch the corresponding IC, and perform
# noise smoothing
for ii in im.iter_ic_indices():
    print ii+1,
    ic = im.get_ic_at_index(ii)
    ic_smooth = window_smooth(ic, window=7)
\end{verbatim}

Although this script is enabled to run in parallel, without special
care taken (or 'mpi4py' installed), it will run in serial mode on
a single CPU:

\begin{verbatim}
$ python proc.py
 -> Reading netCDF file '/x/PyMS/data/gc01_0812_066.cdf'
Size of the intensity matrix is (n_scans, n_mz): 9865 551
1 2 3 4 5 6 7 8 9 10 11 12 13 14 15 16 17 18 19 20 21 22 23 24 25
26 27 28 29 30 31 32 33 34 35 36 37 38 39 40 41 42 43 44 45 46 47
... [ further output deleted ] ...
\end{verbatim}



