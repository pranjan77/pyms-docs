% chapter01.tex

 %%%%%%%%%%%%%%%%%%%%%%%%%%%%%%%%%%%%%%%%%%%%%%%%%%%%%%%%%%%%%%%%%%%%%%%%%%%%%
 %                                                                           %
 %    PyMS documentation                                                     %
 %    Copyright (C) 2005-8 Vladimir Likic                                    %
 %                                                                           %
 %    The files in this directory provided under the Creative Commons        %
 %    Attribution-NonCommercial-NoDerivs 2.1 Australia license               %
 %    http://creativecommons.org/licenses/by-nc-nd/2.1/au/                   %
 %    See the file license.txt                                               %
 %                                                                           %
 %%%%%%%%%%%%%%%%%%%%%%%%%%%%%%%%%%%%%%%%%%%%%%%%%%%%%%%%%%%%%%%%%%%%%%%%%%%%%

\chapter{Introduction}

\section{Chromatography -- mass spectrometry}

PyMS is software for processing of chromatography--mass spectrometry
data. Mass spectrometry is an analytical technique widely used in chemistry
and biology for research, quality control studies, and forensic analysis.
Mass spectrometry analysis is based on detection ions separated by 
their different mass-to-charge (m/z) ratio. The ions can be generated
by a variety of methods (thermal ionization, chemical ionization,
electron impact ionization, etc), and can be ionized atoms or molecular
fragments. The separation by m/z ratio could also be achieved in a
variety of ways (static or dynamic electric or magnetic fields). The
combinations of ionization methods and ion separation methods leads
to a variety of mass spectrometers \cite{gross04}.

Chromatography separation relies on a column packed with a stationary
phase (immobile) over which a mobile phase is pushed. The mobile phase
is either gas (gas chromatography, GC) or liquid (liquid chromatography,
LC). The mobile phase carries the complex mixture to be separated. The
two phases are chosen so that components of the mixture have different
solubilities in each phase. Each component of the mixture is in
equilibrium between the stationary and the mobile phase, determined
by its solubility in the two phases. As a result components of the
mixture travel with different speed through the column, resulting in
the separation of the mixture. 

Joining chromatography separation with mass spectrometer results in
hyphenated mass spectrometry, GC-MS or LC-MS, depending on the nature
of the chromatoraphic separation. In GC-MS and LC-MS settings mass
spectrometry analyzer is located at the outlet of the chromatography
column. As the mobile phase leaves the column (possibly with some
component of the sample mixture) the mass spectrometer records a full
mass spectrum (m/z vs intensity). Recording one such spectrum is called
a "scan", and mass spectrometer operates in repetitive scanning mode.
The separation occurs on the time scale of minutes, and a certain
number of scans is performed per unit time to record mass spectra
continuously. The time between the sample injection and the detection
of an analyte on the detector at the end of the column is called the
retention time.  Retention time is unique for the analyte, i.e. each
compound in a mixture.  GC-MS is suitable for profiling of volatile,
thermally stable metabolites (or metabolites made such by chemical
derivatization), while LC-MS is suitable for profiling of predominantly
polar metabolites \cite{halket05}.

Metabolite profiling refers to either a targeted or an all-inclusive
profiling of low-molecular weight metabolites in biological samples.
Chromatography separation coupled to mass spectrometry detection is
a robust analytical approach used to quantitate hundreds of compounds
in metabolomic studies.

The data produced by GC-MS and LC-MS instruments is three dimensional,
with the elution time dimension, the m/z dimention, and an abitrary
intensity axis. This data is typically viewed when projected on the
time axis, to produce the total ion chromatogram (TIC) where individual
chemical components appear as signal peaks. The principles of GC-MS
and LC-MS data processing are well established. A typical data processing
pipeline may involve noise attenuation, baseline correction, peak
detection, and peak quantitaion (integration).  Overlapping signals
may be discerned by processing ion chromatograms for individual masses
(deconvolution), or two-dimensional data processing methods may be
applied directly on the spectral matrix. In GC-MS mass spectra for
individual time points are often matched against large libraries of
mass spectra for compound identification purposes.

\section{About PyMS}

PyMS is a Python toolkit for processing of chromatography--mass
spectrometry data. The idea behind PyMS is to decouple processing
methods form visualization and the concept of interactive processing.
The purpose of this is to provide a set of components for rapid
development and testing of new processing methods and algorithms,
as well as automated data processing.

PyMS is released as open source, under the GNU Public License
version 2.  

