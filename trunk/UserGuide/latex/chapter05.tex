% chapter05.tex

 %%%%%%%%%%%%%%%%%%%%%%%%%%%%%%%%%%%%%%%%%%%%%%%%%%%%%%%%%%%%%%%%%%%%%%%%%%%%%
 %                                                                           %
 %    PyMS documentation                                                     %
 %    Copyright (C) 2005-8 Vladimir Likic                                    %
 %                                                                           %
 %    The files in this directory provided under the Creative Commons        %
 %    Attribution-NonCommercial-NoDerivs 2.1 Australia license               %
 %    http://creativecommons.org/licenses/by-nc-nd/2.1/au/                   %
 %    See the file license.txt                                               %
 %                                                                           %
 %%%%%%%%%%%%%%%%%%%%%%%%%%%%%%%%%%%%%%%%%%%%%%%%%%%%%%%%%%%%%%%%%%%%%%%%%%%%%

\chapter{Data pre-processing}

\section{Noise smoothing}

The purpose of noise smoothing is to remove high-frequency noise from
data, and thereby increase the contribution of the signal relative to
the contribution of the noise.

\subsection{Window averaging}

\noindent
[ {\em This example is in pyms-test/51a} ]

A simple approach to noise smoothing is moving average window smoothing.
In this approach the window of a fixed size ($2N+1$ points) is moved
across the ion chromatogram, and the intensity value at each point is
replaced with the mean intensity calculated over the window size.
The example below illustrates smoothing of TIC by window averaging.

Load the data and get the TIC:

\begin{verbatim}
>>> andi_file = "/x/PyMS/data/gc01_0812_066.cdf"
>>> data = ANDI_reader(andi_file)
 -> Reading netCDF file '/x/PyMS/data/gc01_0812_066.cdf'
>>> tic = data.get_tic()
\end{verbatim}

Apply the mean window smooting with the 5-point window:

\begin{verbatim}
from pyms.Noise.SavitzkyGolay import window_smooth
tic1 = window_smooth(tic, window=5)
 -> Window smoothing (mean): the wing is 2 point(s)
\end{verbatim}

Apply the median window smooting with the 5-point window:

\begin{verbatim}
>>> tic2 = window_smooth(tic, window=5, median=True)
 -> Window smoothing (median): the wing is 2 point(s)
\end{verbatim}

Apply the mean windows smoothing, but specify the window as
a time string (in this example, 7 seconds):

\begin{verbatim}
>>> tic3 = window_smooth(tic, window='7s')
 -> Window smoothing (mean): the wing is 9 point(s)
\end{verbatim}

Time strings are explained in the Section \ref{sec:time-string}.

\subsection{Savitzky--Golay noise filter}

\noindent
[ {\em This example is in pyms-test/51b} ]

A more sophisticated noise filter is the Savitzky-Golay filter.
Given the data loaded as above, this filter can be applied as
follows:

\begin{verbatim}
>>> from pyms.Noise.SavitzkyGolay import savitzky_golay
>>> tic1 = savitzky_golay(tic)
 -> Applying Savitzky-Golay filter
      Window width (points): 7
      Polynomial degree: 2
\end{verbatim}

In this example the default parameters were used.

\section{Time strings}
\label{sec:time-string}

A time string is specification of time interval, that takes the format
'NUMBERs' or 'NUMBERm' for time interval in seconds or minutes. For
example, these are valid time strings: '10s' (10 seconds) and '0.2m'
(0.2 minutes).

\section{Baseline correction}

\noindent
[ {\em This example is in pyms-test/52} ]

Baseline distortion originating from instrument imperfections and
experimental setup is often observed in mass spectrometry data,
and off-line baseline correction is often an important step in
data pre-processing. There are many approaches for baseline
correction. One advanced approach is based top-hat transform
developed in mathematical morphology \cite{serra83}, and used
extensively in digital image processing for tasks such as image
enhancement. Top-hat baseline correction was previously applied
in proteomics based mass spectrometry \cite{sauve04}.

PyMS currently implements only top-hat baseline corrector, using
the SciPy package 'ndimage'. For this feature to be available either
SciPy (Scientific Tools for Python \cite{scipy}) must be installed,
or the local versions of scipy's ndimage must be installed. For
the SciPy/ndimage installation instructions please see the section
\ref{subsec:scipy-ndmage}.

Application of the top-hat baseline corrector requires the size
of the structural element to be specified. The structural element
needs to be larger than the features one wants to retain in the
spectrum after the top-hat transform. In the example below, the
top-hat baseline corrector is applied to the TIC of the data set
'gc01\_0812\_066.cdf', with the structural element of 1.5 minutes:

\begin{verbatim}
>>> from pyms.GCMS.IO.ANDI.Function import ANDI_reader
>>> andi_file = "/x/PyMS/data/gc01_0812_066.cdf"
>>> data = ANDI_reader(andi_file)
 -> Reading netCDF file '/x/PyMS/data/gc01_0812_066.cdf'
>>> tic = data.get_tic()
>>> from pyms.Noise.SavitzkyGolay import savitzky_golay
>>> tic1 = savitzky_golay(tic)
 -> Applying Savitzky-Golay filter
      Window width (points): 7
      Polynomial degree: 2
>>> from pyms.Baseline.TopHat import tophat
>>> tic2 = tophat(tic1, struct="1.5m")
 -> Top-hat: structural element is 239 point(s)
>>> tic.write("output/tic.dat",minutes=True)
>>> tic1.write("output/tic_smooth.dat",minutes=True)
>>> tic2.write("output/tic_smooth_bc.dat",minutes=True)
\end{verbatim}

\noindent
In the interactive session shown above, the data set if first loaded,
Savitzky-Golay smoothing was applied, followed by baseline correction.
Finally the original, smoothed, and smoothed and baseline corrected
TIC were saved in the directory 'output/'.

