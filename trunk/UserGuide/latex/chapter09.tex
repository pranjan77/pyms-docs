% chapter09.tex

 %%%%%%%%%%%%%%%%%%%%%%%%%%%%%%%%%%%%%%%%%%%%%%%%%%%%%%%%%%%%%%%%%%%%%%%%%%%%%
 %                                                                           %
 %    PyMS documentation                                                     %
 %    Copyright (C) 2005-2010 Vladimir Likic                                 %
 %                                                                           %
 %    The files in this directory provided under the Creative Commons        %
 %    Attribution-NonCommercial-NoDerivs 2.1 Australia license               %
 %    http://creativecommons.org/licenses/by-nc-nd/2.1/au/                   %
 %    See the file license.txt                                               %
 %                                                                           %
 %%%%%%%%%%%%%%%%%%%%%%%%%%%%%%%%%%%%%%%%%%%%%%%%%%%%%%%%%%%%%%%%%%%%%%%%%%%%%

\chapter{Parallel processing with PyMS}

\section{\label{sec:mpi}Requirements}

Using PyMS parallel capabilities requires installation of the package
'mpi4py', which provides bindings of the Message Passing Interface (MPI)
for the Python programming language. This package can be downloaded
from {\tt http://code.google.com/p/mpi4py/}. Since 'mpi4py' provides
only Python bindings, it requires an MPI implementation. We recommend
using mpich2:\\
{\tt http://www.mcs.anl.gov/research/projects/mpich2/}\\
We show the installation of 'mpich2' and 'mpi2py' on Linux system from
software distributions downloaded from the projects' web site.

\subsection{\label{sec:mpich2}Installation of 'mpich2'}

\begin{enumerate}

\item From the mpich2 project web site download the current distribution of
mpich2 (in our case the file 'mpich2-1.2.1p1.tar.gz').

\item Prepare the directory for mpich2 installation. In this example
we have chosen to use /usr/local/mpich2/. Our version of mpitch2 is
1.2.1, and to allow for the installation of different version later,
we create a subdirectory "1.2.1",

\begin{verbatim}
$ mkdir -vp /usr/local/mpich2/1.2.1
\end{verbatim}

The above command will make the directory /usr/local/mpich2/ and also
/usr/local/mpich2/1.2.1. Note that /usr/local is usually owned by
root, and the above commands may require root privileges.

\item Unpack this file and change to the source code directory:

\begin{verbatim}
$ tar xvfz mpich2-1.2.1p1.tar.gz 
$ cd  mpich2-1.2.1p1
\end{verbatim}

\item Configure, compile, and install mpich2:

\begin{verbatim}
$ ./configure --prefix=/usr/local/mpich2/1.2.1 --enable-sharedlibs=gcc
$ make
$ make install
\end{verbatim}

If /usr/local/mpich2/1.2.1 is owned by rood, the above command
may require root privileges.

\end{enumerate}

\subsection{\label{sec:mpi4py}Installation of 'mpi4py'}

\begin{enumerate}

\item From the mpi4py project web site download the current distribution
of mpi4py (in our case the file 'mpi4py-1.2.1.tar.gz').

\item Unpack this file and change to the source code directory:

\begin{verbatim}
$ tar xvfz mpi4py-1.2.1.tar.gz
$ cd mpi4py-1.2.1
\end{verbatim}

\item Edit the file 'mpi.cfg' to reflect the location of mpich2.  In
our case this file after editing contained the following:

\begin{verbatim}
# MPICH2
[mpi]
mpi_dir              = /usr/local/mpich2/1.2.1
mpicc                = %(mpi_dir)s/bin/mpicc
mpicxx               = %(mpi_dir)s/bin/mpicxx
\end{verbatim}

\item Install mpi4py:

\begin{verbatim}
$ python setup.py install
\end{verbatim}

\item Check that mpi4py works:

\begin{verbatim}
$ python
Python 2.5.2 (r252:60911, Sep 10 2008, 14:39:22) 
[GCC 4.1.1 20070105 (Red Hat 4.1.1-52)] on linux2
Type "help", "copyright", "credits" or "license" for more information.
>>> import mpi4py
>>> 
\end{verbatim}

If the above command import produced no output, mpi4py is installed
properly and ready to use.

\end{enumerate}



